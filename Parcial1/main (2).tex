\documentclass{article}
\usepackage[utf8]{inputenc}
\usepackage[spanish]{babel}
\usepackage{listings}
\usepackage{graphicx}
\graphicspath{ {images/} }
\usepackage{cite}

\begin{document}

\begin{titlepage}
    \begin{center}
        \vspace*{1cm}
            
        \Huge
        \textbf{Informe de desarrollo del proyecto}
            
        \vspace{0.5cm}
        \LARGE
        Infoma2 S.A.S.
            
        \vspace{1.5cm}
            
        \textbf{Jose Manuel Rivera.\\
        Julian Perez Lopez.\\
        Jesús Antonio Ibarra.}
            
        \vfill
            
        \vspace{0.8cm}
            
        \Large
        Despartamento de Ingeniería Electrónica y Telecomunicaciones\\
        Universidad de Antioquia\\
        Medellín\\
        Abril de 2021
            
    \end{center}
\end{titlepage}

\tableofcontents
\newpage
\section{Introducción}\label{intro}
¿Qué es Arduino?\\

Arduino es una plataforma de desarrollo basada en una placa electrónica de hardware libre que incorpora un microcontrolador re-programable y una serie de pines.Estos permiten establecer conexiones entre el microcontrolador y los diferentes sensores de una manera muy sencilla.\\

Teniendo esto en cuenta, esta actividad tiene como objetivo poner a prueba las destrezas en análisis de problemas y la capacidad para trabajar con Arduino e integrar la programación en C++, además de usar de manera adecuada las funciones de la plataforma que permiten controlar el puerto serial y los puertos digitales. Para desarrollar la actividad, se debe tener las destrezas y conocimientos fundamentales de la programación con C++, en donde se  resaltara las estructuras de programación como tipos de datos, apuntadores, arreglos y funciones; Las cuales nos permitira usar matrices de leds para imprimir mensajes


\newpage

\section{Contenido} \label{contenido}
Segun los requisitos establecidos por la actividad, se describira el analisis del problema y sus soluciones, un esquema de tareas, el algoritmo, los problemas que se presentaron. 

\subsection{Analisis del problema y sus soluciones}
Un negocio que quiere atraer mas clientes, nos pide generear un letrero que imprime un mensaje usando matrices de leds, con el fin de que los clientes encuentren el negocio con facilidad y lo recuerden, para que asi puedan volver. Este letrero hecho por leds debe funcionar cuando el negocio este abierto o se este ofreciendo algun producto en particular.
\begin{itemize}
    \item Para desarrollar este proyecto se hara uso de la plataforma de Tinkercad, ya que se podra realizar el circuito sin tener que preocurse por riesgo que conlleva un circuito fisico.
    \item Se utilizara una aplicacion de desarrollo de programa llamada QT, que nos permitira la facilidad de hacer el codigo que se implementara en el circuito.
    \item Se investigara sobre las areas que se desconocen usando paginas wed o pidiendo guias a personas experimentadas con el conocimiento suficiente para ayudarnos, en este caso, los profesores.
\end{itemize}

\subsection{Esquema de desarrollo del algoritmo}
\begin{itemize}
    \item Lo primero que se hizo fue entender lo que nos pide la actividad, despues cada quien empieza a buscar ideas de posibles soluciones para la construccion del circuito. Cuando nos reunimos, se empieza a desarrollar el circuito a partir de quien tiene mejores bases para su construccion.
    \item Para la creacion de codigo, cada miembro del equipo empezara su elaboracion hasta que se exponga su correcto funcionamiento.\\ 
    En esta parte es donde se encontrara mas problemas, por eso, cada vez que un miembro encuentre un error los otros lo ayudaran a resolverlo hasta que funcione correctamente.
    \item Se implementara en el algoritmo funciones que nos serviran para verificar que el circuito funciona, ingresar y mostrar una imagen.
\end{itemize}
\subsection{Algoritmo}
//Serial para las 8 filas :7\\
//Timer de salida para todo: 6\\
//Serial para asignar la fila: 2\\
//Timer para asigar la fila:3\\
//Timer por columna:5\\
int verifi(int arr[8][8]);\\
void iMPatron(int arreg[8][8]);\\

const int serFila =2;\\
const int Rfil =3;\\
const int Rsalcol =4;\\
const int RCol =5;\\
const int RSalfil=6;\\

int main(){\\

  // Inicializacion de puertos\\
  for(int pin = 2; pin<7;pin++){\\
  	 pinMode(pin, OUTPUT);\\
  }\\
  Serial.begin(9600);\\
  
  
  byte arr[8]={0xFF,0x7F,0x3F,0x1F,0xF,0x7,0x3,0x1};\\
  byte arr2[8]={0xFF,0xFF,0xFF,0xFF,0xFF,0xFF,0xFF,0xFF};\\
  int patron[8][8];\\
	bool ps =false;\\
    
  while(true){\\
    for(int i = 0; i <8;i++){\\
      for(int bi =0; bi <8;bi++){\\
        if(ps){patron[i][bi]= GetBit(arr[i],bi);}\\
        else{patron[i][bi]= GetBit(arr[i],7-bi);}\\
        Serial.print(patron[i][bi]);\\
      }\\
      Serial.println();\\
    }\\
    Serial.println();\\
    iMPatron(patron);\\

    if(ps){ps=false;}\\
    else{ps=true;}\\
  }\\
  }\\
bool GetBit( byte N, int pos)\\
   {                 // pos = 7 6 5 4 3 2 1 0\\
       int b = N >> pos ;         // Shift bits\\
       b = b & 1 ;                // coger solo el ultimo bit\\
       return b ;\\
   }\\
int verifi(int arr[8][8]){\\

    int i=0;\\
    int j=0;\\
    for (int i=0; i<8; i++){//Recorre todas las posiciones de las filas\\
        for (int j=0;j<8;j++){  //Recorre todas las posiciones de las columnas\\
            	//cout<<"Digite un numero ["<<f<<"]["<<c<<"] :";\\
            	//cin>>arreg[f][c];\\
            }\\
        }\\
    for (int i=0;i<8;i++){ //Imprime la matriz con los led prendidos\\
        for(int j=0;j<8;j++){\\
            //cout<<arr[i][j];\\
        }\\
        //cout<<'\n';\\
    }\\
    return arr[i][j];\\
}\\

void publik(){\\
//# p\\
  int p;\\
  int patrones[p][8][8];\\
  //Ingreso por el serial\\
  for (int patron = 0;patron <p;p++){\\
  	//patrones[patron] = verifi()\\
  } \\
}\\
void iMPatron(int arreg[8][8]){\\
  /*\\
  for(int c=0;c<8;c++){\\
    for(int f =0; f<8;f++){\\
      digitalWrite(serFila,arreg[f][c]);\\
      Pulso(Rfil);\\
      Pulso(RSalfil);\\
      
      Serial.print(arreg[f][c]);\\
      Serial.print(", ");\\
  }\\
  
  Serial.println();\\
  delay(500);\\
    
  Pulso(RCol);\\
  Pulso(Rsalcol);\\
  }\\
	*/\\
  for(int x=0;x<8;x++){\\
    for(int y = 0;y<8;y++){\\
      digitalWrite(serFila,arreg[y][x]);\\
      Pulso(Rfil);\\
      Pulso(RSalfil);\\
      
    }\\
  Pulso(RCol);\\
  Pulso(Rsalcol);\\

  }\\
}\\
void Pulso(int pin){\\
  digitalWrite(pin,0);\\
  digitalWrite(pin,1);\\
  digitalWrite(pin,0); \\
}\\

\subsection{Problemas que se presentaron}


\newpage
\section{Conclusion} \label{conclusion}
Para concluir, aunque no se tiene muchas ideas para el desarrollo de este proyecto, serviran como base para implementar y configurar el juego como mejor parezca, ya que estas ideas bases son de vital importantes para el tipo de juego que se quiere crear y daran bienvenidas a futuros cambios para mejorar, como el entretenimiento para los jugadores y la diversidad de escenas o para corregir fallos y llenar huecos que apareceran mas adelante.

\end{document}
